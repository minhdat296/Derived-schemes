\section{Sheaf cohomology on sites}
    We develop, throughout this section, the theory of sheaf cohomology (in the classical sense\footnote{I.e. we restrict our attention to cohomologies of abelian and (quasi-)coherent sheaves and exclude cohomology theories such as constructible cohomology from our discussions.}) on a general site. 

    \subsection{Abelian sheaf cohomology}
        \subsubsection{Abelian sheaf cohomologies as right-derived functors}
            \begin{convention}
                If $A$ is a (commutative) ring object within some topos, then we shall denote the corresponding \textit{abelian} category of \textit{underived} $A$-modules by $A\mod$, whereas $A\mod$ shall be used for denoting the derived category of $A$-modules with its canonical $t$-structure $(A\mod^{\leq 0}, A\mod^{\geq 0})$. Also, our indexing will always be cohomological! 
            \end{convention}
            \begin{remark}
                Recall that if $(\C, J)$ is a site then the category $\Shv_{\Z}(\C, J)^{\heart}$, whose objects are sheaves of abelian groups on $(\C, J)$, will be an abelian category with enough injectives. As such, any object $\calF \in \Ob(\Shv_{\Z}(\C, J)^{\heart})$ admits an injective resolution $\calF \to \calF^{\bullet}$ wherein $\calF^{\bullet}[0] \cong \calF$ and $\calF^{\bullet}[d]$ is injective for all $d \geq 1$. This phenomenon leads us to the following definition of ($J$-)sheaf cohomology on the site $(\C, J)$.
            \end{remark}
            \begin{convention}
                If $A$ is a sheaf of rings (or even a sheaf of monoids) on a site $(\C, J)$ then we shall write $\Shv_A(\C, J)$ for the category of (sheaves of) $A$-modules on $(\C, J)$. The only exception to this rule is when $A$ is the structure sheaf $\calO_X$ of a ringed site $(X, \calO_X)$: in that case, we shall stick to the traditional notation $\calO_X\mod$ instead of writing $\Shv_{\calO_X}(X)$.
            \end{convention}
            \begin{remark}
                Even though we work with sheaves of abelian groups in what follows, constructing abelian sheaf cohomology theories with values in modules over other (commutative) rings is a simple matter of base-change. 
            \end{remark}
            \begin{definition}[Sheaf cohomology on sites] \label{def: sheaf_cohomology_on_sites}
                Let $(\C, J)$ be a site and let $U \in \Ob(\C)$ be an object thereof. To each such object $U$, one may associate a family $\{H^i_{(\C, J)}(U, -)\}_{i \in \Z}$ of \textbf{cohomology functors}\footnote{Hereinafter, $\calF^{\bullet}$ is to be understood as some choice of an injective resolution of some sheaf of abelian groups $\calF \in \Ob(\Shv_{\Z}(\C, J)^{\heart})$. Again, we are allowed to make such a choice, as $\Shv_{\Z}(\C, J)^{\heart}$ has enough injectives.}:
                    $$H^i_{(\C, J)}(U, -): \Shv_{\Z}(\C_{/U}) \to \Z\mod^{\heart}$$
                    $$\calF^{\bullet} \mapsto H^i_{\Z}(\Gamma(U, \calF^{\bullet}))$$
            \end{definition}
            \begin{remark}[Cohomology functors are exact] \label{remark: cohomology_functors_are_exact}
                Per some basic homological algebra, any family of cohomology functors $\{H^i_{(\C, J)}(U, -)\}_{i \in \Z}$ associated to an object $U$ of a site $(\C, J)$ as in definition \ref{def: sheaf_cohomology_on_sites} determines a functor:
                    $$H^*_{(\C, J)}(U, -): \Shv_{\Z}(\C_{/U}) \to \Z\mod$$
                    $$\calF^{\bullet} \mapsto H^*(\Gamma(U, \calF^{\bullet}))$$
                wherein $H^*(\Gamma(U, \calF^{\bullet}))$ is to be understood as the long exact sequence of cohomology groups associated to the long exact\footnote{Recall that the usual algebraic cohomology functor $H^*(-)$ is exact \textit{a priori} (this is an easy but somewhat tedious homological algebra exercise).} sequence of abelian groups $\Gamma(U, \calF^{\bullet})$. As such, any cohomology functor $H^*_{(\C, J)}(U, -)$ can be interpreted as the composition:
                    $$H^* \circ \R\Gamma(U, -): \Shv_{\Z}(\C_{/U}) \to \Z\mod$$
                Since the functors $\R\Gamma(U, -)$ and $H^*$ are both exact \textit{a priori}, so is the cohomology functor $H^*_{(\C, J)}(U, -)$. 
                
                Furthermore, since both $\R\Gamma(U, -)$ and $H^*$ are right-derived functors, the universal property of derived functors (or rather, that of Kan extensions) guarantees that for any object $U \in \Ob(\C)$, the corresponding cohomology functor $H^*_{(\C, J)}(U, -)$ also arises as the right-derived functor of some (left-)exact functor $\Shv_{\Z}(\C, J)^{\heart} \to \Z\mod^{\heart}$. In particular, this means that the cohomology functor $H^*_{(\C, J)}(U, -)$ is a homological universal $\delta$-functor (in the sense of Grothendieck).
                
                Now, it should be noted that since taking cohomologies is an operation that exists and is well-defined on the derived category of any abelian category, it is the right-derived functor:
                    $$\R\Gamma(U, -): \Shv_{\Z}(\C_{/U}) \to \Z\mod$$
                of the global section functor $\Gamma(U, -): \Shv_{\Z}(\C_{/U}) \to \Z\mod^{\heart}$ that deserves the most attention when abelian sheaf cohomology is discussed. In fact, turning our attention from the cohomology functor $H^*_{(\C, J)}(U, -)$ to the derived global section functor $\R\Gamma$ is also practically beneficial, as doing so simplifies discussion of the compatibility of taking cohomologies with other functors on abelian sheaf categories (e.g. higher direct images).
            \end{remark}
            \begin{remark}[Resolution-independence of cohomology functors] \label{remark: resolution_independence_of_cohomology_functors}
                By combining the fact that the operation of taking cohomologies does not depend on any choice of resolution with the analysis from remark \ref{remark: cohomology_functors_are_exact}, one sees that indeed, the cohomology groups $H^i_{(\C, J)}(U, \calF)$ as in definition \ref{def: sheaf_cohomology_on_sites} are independent of the choice of injective resolution $\calF^{\bullet}$; as such, the cohomology functors $H^*_{(\C, J)}(U, \calF)$ are well-defined as functors on the derived category $\Shv_{\Z}(\C_{/U})^{\heart}$. 
            \end{remark}
            \begin{convention}
                From now on, if the underlying site $(\C, J)$ is clearly understood from the given context, we shall simplify our notations and only write $H^*(U, \calF)$ for abelian sheaf cohomologies.
            \end{convention}
            
        \subsubsection{Interpreting low-dimensional abelian sheaf cohomologies}
            
    \subsection{Cohomology of ringed sites}
        \begin{remark}
            Let $(\C, J)$ be a site. Definition \ref{def: sheaf_cohomology_on_sites} turns out to actually be perfectly adaptable to more general categories of modules over ring objects internal to the sheaf topos $\Sh(\C, J)$ other than the constant ring $\Z$. This is thanks to the fact that much like $\Shv_{\Z}(\C, J)^{\heart}$, any module category $\Shv_A(\C, J)^{\heart}$ of modules over some sheaf of rings $A$ is also an abelian category with enough injectives \textit{a priori}. The subsequent remarks also apply. 
        \end{remark}
        
        \subsubsection{Locality}
            \begin{convention}
                Throughout this subsection, we suppose that every site has terminal objects. In particular, assuming so allows us to \say{restrict} sheaves: if $\C$ is a site (with terminal objects, say $X$) and $U \in \Ob(\C)$ is an object therein, and if $\calF \in \Ob(\Sh(\C))$ is a sheaf on $\C$, then we shall write $\calF|_U$ for the pullback of $\calF$ along the canonical morphism $U \to X$. This is not an overly restrictive assumption, as most sites that appear in algebraic geometry arise naturally from slice categories and as such come naturally equipped with terminal objects. 
            \end{convention}
            
            \begin{proposition}[Injectivity is preserved by restrictions] \label{prop: injectivity_is_preserved_by_restrictions}
                Let $(\C, \calO)$ be a ringed site and let $U \in \Ob(\C)$ be an object thereof. 
                    \begin{enumerate}
                        \item If $\calF \in \Ob(\calO\mod^{\heart})$ is an injective object then $\calF|_U \in \Ob(\calO|_U\mod^{\heart})$ will also be an injective object. 
                        \item In fact, there is a canonical quasi-isomorphism $H^*_{\C}(U, \calF) \overset{\qis}{\cong} H^*_{\C_{/U}}(U, \calF)$.
                    \end{enumerate}
            \end{proposition}
                \begin{proof}
                    \noindent
                    \begin{enumerate}
                        \item This is a straightforward corollary of the fact that additive left-adjoints between abelian categories (in this case, the pullback functor $\calF \mapsto \calF|_U$) preserve injectivity of objects (cf. \cite[\href{https://stacks.math.columbia.edu/tag/015Z}{Tag 015Z}]{stacks}).
                        \item This is a direct consequence of the fact that $\Gamma_{\C}(U, \calF) \cong \Gamma_{\C_{/U}}(U, \calF|_U)$.
                    \end{enumerate}
                \end{proof}
        
        \subsubsection{\v{C}ech complexes}
            \begin{definition}[The \v{C}ech complex] \label{def: the_cech_complex}
                
            \end{definition}