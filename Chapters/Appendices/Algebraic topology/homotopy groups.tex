\section{Homotopy groups of spaces}
    \subsection{The fundamental group(oid) and van Kampen's Theorem}
        \begin{definition}[Homotopies] \label{def: homotopies}
            Consider two continuous functions:
                $$
                    \begin{tikzcd}
                        X \arrow[r, "g"', bend right, shift right] \arrow[r, "f", bend left, shift left] & Y
                    \end{tikzcd}
                $$
            A \textbf{homotopy} between them is then  
        \end{definition}
    
        \begin{definition}[The fundamental ($\infty$-)groupoid] \label{def: the_fundamental_groupoid}
            Let ${}^{*/}\bbS^1$ denote the $1$-sphere with a marked point. Then, for any pointed topological space $(X, x)$, denote by $\Pi_{\infty}(X, x)$ the $\infty$-groupoid wherein:
                \begin{itemize}
                    \item $0$-morphisms are points of $X$,
                    \item $1$-morphisms are marked-point-preserving continuous functions ${}^{*/}\bbS^1 \to (X, x)$ (which we call \textbf{loops based at $x$} in $X$),
                    \item $2$-morphisms are homotopies between these loops,
                    \item etc.
                \end{itemize}
        \end{definition}
    
    \subsection{Higher homotopy groups}
        \subsubsection{Lifting properties}
            \begin{definition}[Lifting problems] \label{def: lifting_problems}
                A \textbf{lifting problem} with solution $h: B' \to E$ is a commutative diagram in an arbitrary category $\C$ of the form:
                    $$
                        \begin{tikzcd}
                        	{E'} & E \\
                        	{B'} & B
                        	\arrow["{p'}"', from=1-1, to=2-1]
                        	\arrow["f", from=2-1, to=2-2]
                        	\arrow["p", from=1-2, to=2-2]
                        	\arrow["\varphi", from=1-1, to=1-2]
                        	\arrow["{\tilde{f}}", dashed, from=2-1, to=1-2]
                        \end{tikzcd}
                    $$
            \end{definition}
            \begin{definition}[Left and right-lifting properties] \label{def: lifting_properties}
                \noindent
                \begin{itemize}
                    \item \textbf{(Right-liftings):} Let $\calF$ be a (small) collection of arrows in some category $\C$. We say that a morphism $p: E \to B$ has the \textbf{right-lifting property} with repsect to $\calF$ if and only if for all $(p': E' \to B') \in \calF$, there will always exist a lifting solution:
                        $$
                            \begin{tikzcd}
                            	{E'} & E \\
                            	{B'} & B
                            	\arrow["{p'}"', from=1-1, to=2-1]
                            	\arrow["f", from=2-1, to=2-2]
                            	\arrow["p", from=1-2, to=2-2]
                            	\arrow["\varphi", from=1-1, to=1-2]
                            	\arrow["{\tilde{f}}", dashed, from=2-1, to=1-2]
                            \end{tikzcd}
                        $$
                    for all arrows $f: B' \to B$ and $\tilde{f}: E' \to E$ making the outer square commute. 
                    \item \textbf{(Left-liftings):} Conversely, we say that a morphism $p: E \to B$ has the \textbf{left-lifting property} with repsect to $\calF$ if and only if for all $(p'': E'' \to B'') \in \calF$, there will always exist a lifting solution:
                        $$
                            \begin{tikzcd}
                            	{E} & {E''} \\
                            	{B} & {B''}
                            	\arrow["{p}"', from=1-1, to=2-1]
                            	\arrow["g", from=2-1, to=2-2]
                            	\arrow["{p''}", from=1-2, to=2-2]
                            	\arrow["\psi", from=1-1, to=1-2]
                            	\arrow["{\tilde{g}}", dashed, from=2-1, to=1-2]
                            \end{tikzcd}
                        $$
                    for all arrows $g: B \to B''$ and $\tilde{g}: E \to E''$ making the outer square commute.
                \end{itemize}
            \end{definition}
            \begin{example}[Lifting properties in the category of topological spaces] \label{example: liftings_of_topological_spaces}
                Within the category of topological spaces and contiunuous functions between them, so-called \textbf{homotopy liftings} are of particular interest, as they can help us study the relationship between homotopy groups between base spaces and their covering spaces.
                
                Let $X$ be an arbitrary topological space. A continuous map $p: E \to B$ has the \textbf{homotopy right-lifting property} (also known as the \textbf{homotopy lifting property}) with respect $X$ if it has the right-lifting property with respect to the singleton set of maps $\{\delta_0: X \to X \x [0, 1]\}$, where $\delta_0$ is given by $\delta_0(x) \mapsto (x, 0)$ for all $x \in X$. Likewise, a continuous map $p: E \to B$ has the \textbf{homotopy left-lifting property} (also known as the \textbf{homotopy extension property}) with respect $X$ if it has the left-lifting property with respect to the singleton set of maps $\{\delta_0: X \to X \x [0, 1]\}$, where $\delta_0$ is as above. Note that in both these situations, one is literally lifting homotopies $\eta: X \x [0, 1] \to B$ to homotopies $\tilde{\eta}: X \x [0, 1] \to E$ along the given map $p: E \to B$. One usually requires that the map $p: E \to B$ is surjective as well.
                
                Within a category of sufficiently convenient topological spaces (e.g. compactly generated topological spaces, $\Delta$-generated topological spaces and everything stronger such as path-connected spaces, etc.) - which is to say that such a category is a closed (symmetric) monoidal with repsect to products - every map $X \to X \x [0, 1]$ corresponds to a unique map $X^{[0, 1]} \to X$ (i.e. to a map from the space of continuous paths $\gamma: [0, 1] \to X$ to $X$ itself)\footnote{For this reason, homotopy extension properties are also known as path-lifting properties}. In particular, the map $\delta_0: X \to X \x [0, 1]$ as above corresponds to the path  
            \end{example}
    
        \subsubsection{Quasi-fibrations and long exact sequences of homotopy groups}
            \begin{definition}[Homotopy groups] \label{def: homotopy_groups}
                
            \end{definition}
            
            \begin{definition}[Quasi-fibration] \label{def: quasi_fibration}
                A \textbf{quasi-fibration} is nothing but an $(\infty, 1)$-kernel in the $(\infty, 1)$-category of \textit{pointed} $\infty$-groupoids, i.e. an $(\infty, 1)$-pullback of the form:
                    $$
                        \begin{tikzcd}
                        	X & \pt \\
                        	Y & Z
                        	\arrow[from=1-1, to=2-1]
                        	\arrow[from=2-1, to=2-2]
                        	\arrow[from=1-2, to=2-2]
                        	\arrow[from=1-1, to=1-2]
                        	\arrow["\lrcorner"{anchor=center, pos=0.125}, draw=none, from=1-1, to=2-2]
                        \end{tikzcd}
                    $$
            \end{definition}
            
            \begin{theorem}[Induced long exact sequence of homotopy groups] \label{theorem: induced_long_exact_sequence_of_homotopy_groups}
                
            \end{theorem}
                \begin{proof}
                    
                \end{proof}
        
        \subsubsection{The Mayer-Vietoris Sequence}