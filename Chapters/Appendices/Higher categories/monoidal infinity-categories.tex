\section{Monoidal \texorpdfstring{$(\infty, 1)$}{}-categories and \texorpdfstring{$(\infty, 2)$}{}-categories}
    \subsection{Monoidal \texorpdfstring{$(\infty, 1)$}{}-categories}
        \begin{convention}
            Until the end of this subsection, the term \say{$\infty$-category} shall be used in place of \say{$(\infty, 1)$-category}, barring extreme prejudices.
        \end{convention}
    
        \subsubsection{Monoidal structures; algebra objects}
            Monoidal categories (cf. definition \ref{def: monoidal_categories}) are categories $\V$ equipped with a bifunctor $\tensor: \V \x \V \to \V$ satisfying axioms which help us ensure that (finite) \say{tensor products} of objects in $\V$ with one another would behave in a similar manner to tensor products of modules. Notably, these \say{tensor products} are only defined up to isomorphisms; in other words, one says that the bifunctor $\tensor$ is subjected to certain \say{coherence conditions}. Now, what if we would like a theory of monoidal categories wherein said coherence conditions admit coherence conditions of their own, e.g. isomorphisms between say, $x \tensor y$ and $y \tensor x$ form not a set\footnote{Let us ignore set-theoretic issues for now.} but rather, a groupoid whose objects are these isomorphisms and whose (invertible) $1$-cells are \say{$2$-isomorphisms} between these isomorphisms (e.g. concretely, one might think of the $2$-category $(1\-\Cat_2, \x, \pt)$ of small $1$-categories with its canonical monoidal structure given by finite products: products $\C \x \D$ are defined up to equivalences of categories, but one may also speak of invertible natural transformations between these equivalences of categories) ? Of course, one can write down these higher coherence conditions by hand, but given the complexity of the definition of monoidal $1$-categories alone (cf. definition \ref{def: monoidal_categories}), it might prove prudent to seek out an alternative method. Moreover, given how $\infty$-categories are \say{spaces} in which equivalences are only specified up to an infinite number of higher coherence constraints, simply trying to specify these coherence conditions might not be so wise a thing to attempt after all.
            
            To that end, let us begin our construction of monoidal $\infty$-categories with an observation.
            \begin{remark}[Monoidal structures as op-fibrations] \label{remark: monoidal_op_fibrations}
                Recall how for $R$ a commutative ring, the binary tensor product $M \tensor_R N$ of two $R$-modules $M$ and $N$ is defined to be the universal $R$-module such that for every pair of $R$-bilinear maps $M \x N \to P$ and $M \x N \to M \tensor_R N$, there exists a unique $R$-linear map $M \tensor_R N \to P$ making the following diagram commute:
                    $$
                        \begin{tikzcd}
                        	{M \x N} & {M \tensor_R N} \\
                        	& P
                        	\arrow[from=1-1, to=1-2]
                        	\arrow[dashed, from=1-2, to=2-2]
                        	\arrow[from=1-1, to=2-2]
                        \end{tikzcd}
                    $$
                One would then show that binary tensor products are nothing special: in fact, one can re-iterate the construction an arbitrary (but finite!) number of times to get $n$-ary tensor products $\bigotimes_{i = 1}^n M_i$ that satisfy a similar kind of universal property; one can also show that the $0$-fold tensor product exists as the zero module. 
                
                At this point, we might suspect that the endowment of finite tensor products to some category $\V$ might be some functorial process $\simp^{\op} \to \V^{\tensor}$ whereby one associates to some given enumerating sequence of indices $[n] \in \simp^{\op}$ a sequence $\{x_1, x_2, ..., x_n\} \in \V^{\tensor}$ of objects of $\V$, which should be thought of as diagrams $[n] \to \V$.
            \end{remark}
            
            We have thus come to the notion of monoidal ($1$-)op-fibrations:
            \begin{definition}[Monoidal op-fibrations] \label{def: monoidal_op_fibrations} \index{Monoidal categories! as op-fibrations}
                Let $(\V, \tensor, \1)$ be a monoidal category. Then, let us associate to it a $1$-functor:
                    $$\V^{\tensor}: \simp^{\op} \to 1\-\Cat_1$$
                subjected to the following constraints:
                    \begin{enumerate}
                        \item $\V^{\tensor}$ ought to preserve the terminal object $[0] \in \simp^{\op}$, i.e.:
                            $$\V^{\tensor}([0]) \cong \pt$$
                        \item For all $n \geq 1$, the following natural $n$-tuple of natural arrows in $\simp^{\op}$:
                            $$[n] \to [1]: \left((i \to i + 1) \mapsto (0 \to 1)\right) \: (\forall i \in [n])$$
                        defines an equivalence:
                            $$\V^{\tensor}([n]) \cong \V^{\tensor}([1])^n$$
                    \end{enumerate}
            \end{definition}
            The functor $\V^{\tensor}$ shall - in time - be thought of as the monoidal structure on $\V$. This, however, is a somewhat non-trivial fact, along with the following, which establishes some descent-theoretic coherence properties of monoidal structures as in definition \ref{def: monoidal_op_fibrations}. 
            \begin{proposition}
                Let $(\V, \tensor, \1)$ be a monoidal category and let us assume the Axiom of Choice (which is equivalent to every fibred category having a cleavage; cf. \cite[Definition 3.9]{vistoli_descent}). Then, the canonically associated functor (as in definition \ref{def: monoidal_op_fibrations}):
                    $$\V^{\tensor}: \simp^{\op} \to 1\-\Cat_1$$
                defines a coCartesian fibration:
                    $$p^{\tensor}: 1\-\Cat_1 \to \simp^{\op}$$
                and vice versa.
            \end{proposition}
                \begin{proof}
                    Fix a category $V_n \in 1\-\Cat$ such that $p^{\tensor}(V_n) \cong [n]$ (i.e. a section over $[n] \in \simp^{\op}$) along with a pair of arbitrary arrows $\varphi: V_n \to V$ and $\psi: V_n \to V'$ and inducing the following commutative diagram in $\simp^{\op}$ wherein $h: p^{\tensor}(V) \to p^{\tensor}(V')$ is arbitrary:
                        $$
                            \begin{tikzcd}
                                p^{\tensor}(V') \arrow[rd, "h"'] \arrow[rrd, "p^{\tensor}(\psi)", bend left] &                                                  &       \\
                                                                                                             & p^{\tensor}(V) \arrow[r, "p^{\tensor}(\varphi)"] & {[n]}
                            \end{tikzcd}
                        $$
                    The definition of the category $\simp$ can then be made use of, specifically to realise that $p^{\tensor}(V) \cong [m]$ and $p^{\tensor}([m'])$ for $m, m' \in \N$ such that $m' \geq m \geq n$. Definition \ref{def: monoidal_op_fibrations} and the universal property of products (note that because $p^{\tensor}(V) \cong [m]$ and $p^{\tensor}([m'])$, we have $V \cong \V^{\tensor}([1])^m$ and $V' \cong \V^{\tensor}([1])^{m'}$) then ensures that there exists an arrow $\eta: V' \to V$ making the following diagram commute:
                        $$
                            \begin{tikzcd}
                                V' \arrow[rd, "\eta"', dashed] \arrow[rrd, "\psi", bend left] &                                                  &       \\
                                                                                                             & V \arrow[r, "\varphi"] & {V_n}
                            \end{tikzcd}
                        $$
                        
                    The converse assertion comes directly from the fact that given the Axiom of Choice, every fibred category admits a cleavage.
                \end{proof}
            
        \subsubsection{Monoidal \texorpdfstring{$(\infty, 1)$}{}-categories as coCartesian fibrations}
        
        \subsubsection{Universal constructions in monoidal \texorpdfstring{$(\infty, 1)$}{}-categories}
    
    \subsection{\texorpdfstring{$(\infty, 2)$}{}-categories}